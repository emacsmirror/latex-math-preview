\documentclass{article}
\usepackage{amsmath}

\begin{document}

$ax^2 + bx + c = 0$

$$ax^2 + bx + c = 0$$

\[
 ax^2 + bx + c = 0
\]

\begin{math}
 ax^2 + bx + c = 0
\end{math}

\begin{displaymath}
 ax^2 + bx + c = 0
\end{displaymath}

\begin{equation}
 ax^2 + bx + c = 0
\end{equation}

\begin{gather}
 ax^2 + bx + c = 0
\end{gather}

\begin{align}
 ax^2 + bx + c = 0
\end{align}

\begin{alignat}{2}
 ax^2 + bx + c = 0 \\
 ax^2 + bx + c = 0
\end{alignat}

\begin{equation}
 A = \begin{pmatrix}
      a_{11} & \dots & a_{1n} \\
      \hdotsfor{3} \\
      a_{m1} & \dots & a_{mn}
     \end{pmatrix}
\end{equation}

\c{o} \LaTeX

% invalid LaTeX mathematical expression
\begin{equation}
 A = \begin{pmatrix}
      a_{11} & \dots & a_{1n}
      \hdotsfor{3} \\
      a_{m1} & \dots & a_{mn}
     \end{pmatrix}
\end{equation}

\begin{table}[h]
\begin{center}
\begin{tabular}{|l|c|r|}
\hline
** & ** & ** \\ \hline
** & ** & ** \\ \hline
** & ** & ** \\
\hline
\end{tabular}
\caption{the Moore neighborhood of a}
\end{center}
\end{table} 

\end{document}
